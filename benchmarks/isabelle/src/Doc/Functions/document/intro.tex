\section{Introduction}

Starting from Isabelle 2007, new facilities for recursive
function definitions~\cite{krauss2006} are available. They provide
better support for general recursive definitions than previous
packages.  But despite all tool support, function definitions can
sometimes be a difficult thing. 

This tutorial is an example-guided introduction to the practical use
of the package and related tools. It should help you get started with
defining functions quickly. For the more difficult definitions we will
discuss what problems can arise, and how they can be solved.

We assume that you have mastered the fundamentals of Isabelle/HOL
and are able to write basic specifications and proofs. To start out
with Isabelle in general, consult the Isabelle/HOL tutorial
\cite{isa-tutorial}.



\paragraph{Structure of this tutorial.}
Section 2 introduces the syntax and basic operation of the \cmd{fun}
command, which provides full automation with reasonable default
behavior.  The impatient reader can stop after that
section, and consult the remaining sections only when needed.
Section 3 introduces the more verbose \cmd{function} command which
gives fine-grained control. This form should be used
whenever the short form fails.
After that we discuss more specialized issues:
termination, mutual, nested and higher-order recursion, partiality, pattern matching
and others.


\paragraph{Some background.}
Following the LCF tradition, the package is realized as a definitional
extension: Recursive definitions are internally transformed into a
non-recursive form, such that the function can be defined using
standard definition facilities. Then the recursive specification is
derived from the primitive definition.  This is a complex task, but it
is fully automated and mostly transparent to the user. Definitional
extensions are valuable because they are conservative by construction:
The \qt{new} concept of general wellfounded recursion is completely reduced
to existing principles.




The new \cmd{function} command, and its short form \cmd{fun} have mostly
replaced the traditional \cmd{recdef} command \cite{slind-tfl}. They solve
a few of technical issues around \cmd{recdef}, and allow definitions
which were not previously possible.

In addition there is also the \cmd{partial\_function} command
(see \cite{isabelle-isar-ref}) that supports the definition of partial
and tail recursive functions.


